% This is definitely *NOT* a thesis
\documentclass[14pt,a4paper]{article}
\usepackage[italian]{babel}
\usepackage[utf8]{inputenc}
\usepackage{graphicx}
\usepackage{lmodern}
\usepackage{textcomp}
\usepackage{url}
\usepackage[hidelinks]{hyperref}
\usepackage{pgfplots}
\usepackage[big,binding=0cm]{layaureo}
\author{Alberto Azzalini}
\title{Valutazione dei servizi ecosistemici in ambiente alpino:\\ problemi e prospettive}
\date{26 luglio 2016}

\begin{document}
	\maketitle
	\tableofcontents
	\begin{abstract}
		L'introduzione del concetto di servizi ecosistemici ha permesso di iniziare a prendere in considerazione la possibilità di assegnare un valore economico alle risorse naturali e ai benefici che esse apportano alla società umana: collegare in termini monetari aspetti biofisici e benessere dell'uomo potrebbe rivelarsi un passo decisivo per mettere chiaramente in luce la dipendenza delle società umane dagli ecosistemi naturali.
		Nel passato, spesso non si è riusciti a valutare in modo adeguato l'importanza degli ecosistemi, che sono stati spesso ritenuti un bene di tutti e il cui valore è stato non di rado sottostimato, quando non addirittura ritenuto trascurabile.
		
		La perdita dei servizi forniti dagli ecosistemi naturali comporterà la necessità di trovare alternative dispendiose. Gli investimenti nel nostro capitale naturale consentiranno invece verosimilmente di risparmiare nel lungo periodo e per questo sono essenziali per il nostro benessere e per la sopravvivenza.
		
		Nell'ambiente alpino, i servizi ecosistemici autoctoni hanno un ruolo importante sia per l'area sulla quale fisicamente insistono e per i territori immediatamente vicini, sia per l'intera Europa: l'arco alpino infatti alimenta i bacini idrografici di tutta la parte centrale del continente e le conseguenze delle perturbazioni al loro equilibrio potrebbero avere ripercussioni di proporzioni enormi.
		
		Purtroppo tale equilibrio è già compromesso, come testimonia, tra gli altri, il fenomeno del ritiro dei ghiacciai che, negli ultimi anni, si è accentuato in maniera preoccupante con conseguenze che, per quanto evidenti, non sempre vengono correlate in maniera corretta con i fenomeni legati alle grandi masse ghiacciate alpine.
		
		
	\end{abstract}
	
	\section{I servizi ecosistemici}
		Il Millennium Ecosystem Assessment\footnote{La Valutazione degli ecosistemi del millennio, abbreviato generalmente in MA dall'inglese \textbf{M}illennium Ecosystem \textbf{A}ssessment, è un progetto di ricerca che ha cercato di identificare i cambiamenti subiti dagli ecosistemi e di sviluppare degli scenari per il futuro, basandosi sul trend dei cambiamenti. È stato lanciato nel 2001 con il supporto delle Nazioni Unite ed è costato 24 milioni di dollari.} \cite{MEA_EcosystemsAndHumanWellBeing:Synthesis} ha definito i servizi ecosistemici come ``i benefici multipli forniti dagli ecosistemi al genere umano'', distinguendoli in quattro grandi categorie:
		\begin{itemize}
			\item supporto alla vita (ciclo dei nutrienti, formazione del suolo e produzione primaria);
			\item approvvigionamento (produzione di cibo, acqua potabile, materiali o combustibile);
			\item regolazione (del clima e delle maree, depurazione dell'acqua, impollinazione e controllo delle infestazioni);
			\item valori culturali (estetici, spirituali, educativi e ricreativi).
		\end{itemize}
		
		
		Ovviamente, per quanto il Millennium Assessment enfatizzi l'interconnessione tra gli ecosistemi e il benessere dell'uomo, viene anche riconosciuto che le azioni dell'uomo che influenzano l'ambiente hanno effetti che non riguardano solo il benessere umano ma anche il valore intrinseco\footnote{ Il valore intrinseco di un bene è tale se considerato in se stesso e per se stesso, senza tenere in considerazione la sua utilità per qualuno o qualcos'altro.} delle specie e degli ecosistemi stessi.
		
		Nella presente esposizione, ci si concentrerà sul servizio ecosistemico di tipo \textit{provisioning} dell'approvvigionamento idrico. 
		
		Le Alpi infatti contribuiscono in maniera significativa ai bacini idrografici: le riserve idriche di gran parte del centro Europa, i cui principali fiumi vengono direttamente o indirettamente alimentati dall'acqua che scende dalle cime alpine, dipendono evidentemente dall'apporto idrico dell'arco alpino settentrionale. 
	
	\section{Monetizzare il valore di un ecosistema}

	Prima di approfondire il caso specifico in esame, è opportuno fare una breve introduzione sul concetto di servizio ecosistemico e di valore economico dello stesso.
	
	Avere una buona dotazione di servizi ecosistemici (SE) significa avere una maggior ``ricchezza'' pro capite in termini di capitale naturale, ma anche una minore vulnerabilità, una maggiore salute e resilienza dei territori.
	
	Ecosistemi sani possono offrire un contributo significativo, proprio perché i loro servizi, gratuitamente utilizzati dall'uomo, che costituiscono risorse non sostituibili con quelle del capitale antropico, rappresentano un importante fattore economico, attualmente ignorato dall'economia tradizionale perché senza mercato, ma di importanza strategica in un'ottica di ecologia economica e di nuovi indicatori di integrazione del PIL \cite{oltre_il_PIL}.
	
	L'economia ecologica suggerisce un nuovo approccio per pesare le risorse di un territorio e per riequilibrare i sistemi economici. Dal capitale inteso in senso generico e tradizionale viene enucleato quello ``naturale'', che --naturalmente-- fornisce servizi, mantenendo la stabilità ecologica dei sistemi, valorizzando i territori ricchi di SE e le attività economiche compatibili che concorrono a mantenerne intatte le funzionalità \cite{SE_e_Sostenibilita}. 
	
	La valutazione ecologico-economica ha lo scopo di stimare i SE in termini monetari, al fine di fornire una metrica comune	attraverso cui i benefici dei diversi servizi forniti dagli ecosistemi possano essere quantificati \cite{MEA_EcosystemsAndHumanWellBeing:Synthesis} per permettere di supportare strategie di sostenibilità e di perequazione territoriale, anche a fronte dei cambiamenti globali nel breve, medio e lungo periodo. 
	
	È importante quindi valutare il valore economico totale (TEV) (\cite{freeman2014measurement}; \cite{merlo2005valuing}) delle risorse e dei servizi considerati anche come beni pubblici, tenendoli presenti nelle analisi costi-benefici e nelle valutazioni del danno ambientale. Il TEV costituisce il background metodologico delle valutazioni dei beni ambientali alla cui base c'è la distinzione tra le due grandi categorie di benefici che una risorsa naturale offre: i valori d'uso\footnote{la capacità (numericamente quantificabile come per il valore di scambio) di un bene o di un servizio di soddisfare un dato fabbisogno, o tout-court il valore di utilità.} e i valori di non-uso\footnote{Il valore di non uso di un bene corrisponde al beneficio prodotto dallo stesso bene come conseguenza di predisposizioni e comportamenti del soggetto economico non direttamente legate all'uso del bene. È definibile come la quantità di moneta che si è disposti a pagare per avere l'opportunità di fruizione del bene per sé o per altri o per l'esistenza stessa del bene, oppure la maggiore quantità di moneta che una persona è disposta a pagare per il bene, rispetto all'ammontare  di moneta effettivo che deve pagare per avere l'opportunità dell'uso del bene o perché l'esistenza  del bene stesso sia comunque assicurata.}.  
	
	Conoscere il valore economico totale delle risorse e dei beni ambientali è quindi importante per verificare la razionalità delle scelte di sviluppo, per dare un valore alle politiche di tutela dell'ambiente e individuare le regioni più fragili dove il cambiamento è più probabile. 
	
	Gli attuali strumenti di pianificazione, pur nelle significative differenze, partono da un'analisi dello status delle risorse ambientali, spesso trascurando i processi ecosistemici, le interazioni dinamiche e di controllo dei processi stessi, in particolare le loro relazioni con i fattori economici e sociali. 
	
	Il paradigma dei SE può costituire quindi la base per una revisione dei termini economici con cui considerare il territorio e i suoi capitali attraverso una pianificazione territoriale più consapevole del significato dei processi ecologici e più orientata verso una sostenibilità concreta e durevole.
	
	
	\section{I cambiamenti climatici per l'ecosistema alpino}
	
	Le Alpi presentano in anteprima le sfide con cui dovranno misurarsi gli ecosistemi, gli habitat e le popolazioni in tutta Europa e nel mondo \cite{LeAlpi}.
	
	Negli ultimi cento anni, il clima globale e regionale ha subito cambiamenti drammatici, percepibili da ognuno di noi. In questo arco di tempo, la temperatura è aumentata in media di circa 0,8\,°C a livello globale mentre, nell'arco alpino settentrionale, negli ultimi 30 anni, si è accresciuta addirittura sino a 1,6\,°C. 
	
	Alcuni studi stimano questi aumenti in ben 1\,°C a livello medio mondiale e 2\,°C per il solo arco alpino settentrionale \cite{LeAlpi}.
	
	Per quanto riguarda gli effetti dei cambiamenti climatici, riveste un'importanza ancora più rilevante lo \textbf{slittamento stagionale delle precipitazioni}, con valori inferiori in estate e maggiori alla fine dell'inverno e in primavera.
	
	L'attuale dibattito si concentra spesso solamente sulle variazioni delle temperature. Tuttavia, l'andamento spazio-temporale delle precipitazioni ha conseguenze non meno rilevanti di quello delle temperature. Infatti le precipitazioni nell'ambiente alpino:
		
	\begin{itemize}
		\item sono responsabili dei maggiori danni osservati, provocati da alluvioni, siccità, valanghe e altre calamità causate da eventi meteorologici estremi;
		\item determinano la disponibilità d'acqua e quindi influiscono direttamente sull'agricoltura e sull'economia forestale;
		\item determinano la distribuzione e il tipo di vegetazione e di ecosistemi presenti in un dato territorio;
		\item hanno influenza decisiva sullo sviluppo turistico delle aree, a causa dell'innevamento.
	\end{itemize}
	
	Lo slittamento in avanti delle precipitazioni, che quindi diventano di carattere piovoso anche in alta quota a causa delle temperature maggiori, ha effetti devastanti se si considera che la pioggia, a differenza della neve, a meno di una minima parte che viene trattenuta dal terreno, defluisce direttamente a valle: nei casi estremi, che purtroppo sembrano avere tempi di ritorno sempre più brevi se non brevissimi, la quantità di acqua che raggiunge le valli e i corsi d'acqua principali è tale da mettere a dura prova i sistemi di canalizzazione e di governo dei torrenti che sono dimensionati per un altro tipo di eventi meteorologici.
	
	\subsection{L'erosione del suolo}
	Tralasciando gli eventi estremi, l'aumento delle piogge rispetto alle nevicate contribuisce in maniera sensibile al dilavamento e all'erosione dei pendii, depauperando il suolo delle aree interessate.
	
	Inutile dire che tali fenomeni erosivi hanno pesantissime ripercussioni sugli ecosistemi ivi presenti, con conseguenze rilevanti per attività umane, che sono da un lato interessate da colate di fango e detriti (dovute peraltro non solo alle piogge ma anche allo scioglimento dei ghiacciai) ma anche da un assottigliamento dello strato di humus in zone, quelle montane, in cui esso è di per sé ridotto.
	
	La perdita del sottile strato di humus, a sua volta, provoca perdite dei raccolti e danneggia strade, edifici e altri beni reali. Inoltre un suolo eroso immagazzina meno acqua e inaridisce più velocemente. Di conseguenza, le piante crescono in condizioni peggiori e si innesta così una spirale negativa che implica anche un calo della fertilità del suolo con conseguente diminuzione dei raccolti. Infine, l'erosione del suolo su grandi superfici può modificare negativamente e a lungo termine anche l'immagine del paesaggio.
	 	
	In sintesi, l'erosione innesca un circolo vizioso le cui conseguenze si ripercuotono su:
	
	\begin{description}
		\item[fertilità del suolo:]
		perdita di humus e sostanze nutritive;
		\item[spessore del terreno:]
		perdita di spazio disponibile per la crescita delle radici;
		\item[capacità di ritenzione d'acqua del suolo:]
		i terreni inaridiscono più velocemente;
		\item[costi per l'economia alpestre:]
		diminuzione dei raccolti, risoluzione dei danni;
		\item[costi pubblici:]
		pulizia di strade, sentieri e infrastrutture;
		\item[corsi d'acqua:]
		deposizione di materiale terroso e sostanze nutritive;
		\item[paesaggio:]
		modifica del paesaggio.
	\end{description}
		
	Questo fenomeno di erosione preoccupa già seriamente le amministrazioni più sensibili: alcuni cantoni svizzeri (San Gallo, Glarona, Appenzello Esterno e Appenzello Interno) hanno commissionato uno studio \cite{Bodenerosion_2016-07-07} sull'erosione, con accento su quella dovuta allo sfruttamento dei terreni per il pascolo. Tale studio, evidenziando come l'aumento delle precipitazioni a carattere piovoso in quota sia aumentato, ha fissato delle linee guida per l'agricoltura al fine di ridurre e prevenire il depauperamento del suolo, chiarendo i rischi ai quali si andrebbe in contro qualora l'attuale \textit{trend} erosivo non venisse in qualche modo rallentato e spiegando l'intera questione in un \textit{pamphlet} divulgativo rivolto alla popolazione.
	
	\subsection{L'impatto del riscaldamento globale sui ghiacciai alpini}
		Certamente uno degli effetti più evidenti del riscaldamento globale e, come visto, delle variazioni del regime delle precipitazioni  per quanto riguarda l'arco alpino è la riduzione e il ritiro dei ghiacciai.
		
		Prendendo in considerazione il più grande ghiacciaio dell'Alto Adige, il Malavalle, e il suo vicino, la Vedretta Pendente, entrambi posti in fondo alla Val Ridanna, è possibile apprezzare la drastica riduzione della loro superficie dalle prime osservazioni attendibili (attorno al 1850, fine della cosiddetta Piccola Età Glaciale) fino ad oggi. \cite{monteneve}
		
		Le montagne di norma si trasformano lentamente, ma leggendo i dati (esemplificativi) di cui al diagramma di figura \ref{diag:ridanna}, è chiaro che il clima alpino è mutato in
		maniera significativa nel corso dell'ultimo secolo. L'aumento delle temperature registrato nelle regioni alpine, come visto, è il doppio della media
		globale. Di conseguenza, i ghiacciai alpini si sciolgono. Dal 1850 hanno perso circa la metà del loro volume di
		ghiaccio, e i tassi di perdita hanno subito una forte accelerazione a partire dalla metà degli anni Ottanta.
		Il limite delle nevi perenni si sta alzando e anche l'andamento delle precipitazioni (pioggia, neve, grandine e
		nevischio) cambia. Un gran numero di ghiacciai di piccole e medie dimensioni probabilmente scomparirà entro la
		prima metà del secolo. Si stima che nelle regioni in cui attualmente si verificano precipitazioni nevose, la neve sarà
		sempre più sostituita da piogge invernali, il che ridurrà il numero di giorni di copertura nevosa. 
		
		Ciò incide sul modo in cui le montagne accumulano e immagazzinano acqua in inverno e la ridistribuiscono nei mesi estivi più caldi. Si prevede quindi un aumento del ruscellamento in inverno e una diminuzione durante l'estate, con conseguenze che vanno dall'erosione (come già visto) alla siccità nei mesi estivi.
		
		
		\begin{figure}[p!]
		\begin{tikzpicture}
		\begin{axis}[
		 title style={at={(0.5,0)},anchor=south,yshift=-60, font=\itshape},
		/pgf/number format/set thousands separator = ,
		enlarge x limits=-1,
		width=\linewidth,
%		title={Superficie del Ghiacciaio di Malavalle e della Vedretta Pendente dal 1850 ad oggi},
		ylabel={Superficie [$km^2$]},
		xmin=1825, xmax=2025,
		ymin=0, ymax=15,
		xtick={1850,1875,1900,1925,1950,1975,2000},
		%ytick={1,...,15},
		legend pos=north west,
		ymajorgrids=true,
		minor tick num=5,
		grid style=dashed,
		]
		
		\addplot[
		color=blue,
		mark=square,smooth
		]
		coordinates {
			(1850, 12.4) (1888, 10.7) (1929,10.3) (1958, 8.7) (1983, 9.4) (1997, 7.9) (2006, 7.3) (2015, 6.2)
		};
		\addlegendentry{Ghiacciaio di Malavalle}
		\addplot[
		color=red,
		mark=*,smooth
		]
		coordinates {
			(1850, 2.6) (1888, 2.3)(1929, 1.6) (1958, 1.3) (1983, 1.4) (1997, 1.2) (2006,1)
		};
		\addlegendentry{Vedretta Pendente}
		
tt		\end{axis}
		\end{tikzpicture}
		\caption{Superficie del Ghiacciaio di Malavalle e della Vedretta Pendente dal 1850 ad oggi}
		\label{diag:ridanna}
	\end{figure}
		
	\subsection{Effetti secondari}
		L'impatto dei cambiamenti climatici sui servizi ecosistemici alpini non si limita agli effetti sulla disponibilità di acqua potabile \cite{LeAlpi}. Per ogni grado di aumento della temperatura, il livello della neve si alza di circa 150 metri. Di conseguenza, si accumula meno neve a bassa quota. Quasi la metà delle stazioni sciistiche in Svizzera, e un numero ancora maggiore in Germania, Austria e nei Pirenei, in futuro avranno difficoltà ad attirare turisti e amanti degli sport invernali.
		
		Per non andare troppo lontano, le ultime due stagioni invernali hanno visto l'arrivo della neve utile agli impianti sciistici solo dopo l'inizio di gennaio in quasi tutti i comprensori. Per contro, la primavera è cominciata con nevicate il cui apporto, anche a causa delle alte temperature immediatamente successive alle precipitazioni, è stato poco significativo nell'economia complessiva del ciclo idrologico.
		
		Rimanendo sull'aspetto ludico-turistico, i rinomati villaggi di igloo realizzati in zone considerate particolarmente adatte allo scopo vengono ultimati sempre più tardi: sullo Zugspitze, ad una altitudine di oltre 2600 metri e nell'area del ghiacciaio dello Schneeferner, negli ultimi anni il villaggio degli igloo è stato aperto solo il 31 dicembre.
		
		Anche le specie vegetali migrano verso nord e altitudini maggiori. Le cosiddette ``specie pionieristiche'' si spostano in altezza. Le piante che si sono adattate al freddo vengono ora cacciate dai loro habitat naturali. Entro la fine del XXI secolo le specie vegetali europee potrebbero spostarsi centinaia di chilometri a nord e il 60 per cento delle specie vegetali montane rischia l'estinzione.
		
		La riduzione osservata e prevista del \textit{permafrost} verosimilmente aumenterà le calamità naturali e i danni alle infrastrutture presenti in alta quota. L'ondata di calore che ha colpito l'Europa nel 2003 ha evidenziato le conseguenze potenzialmente gravi dell'aumento delle temperature e della siccità sul benessere umano e sui settori economici che si basano sull'uso di acqua (come la produzione di elettricità). In quel solo anno lo scioglimento dei ghiacci ha ridotto di un terzo la massa dei ghiacciai alpini e ha provocato decine di migliaia di morti in Europa.
	
	\section{Prospettive: l'adattamento}
		Banalmente, la soluzione ai problemi dell'arco alpino e, più in generale, del riscaldamento che coinvolge tutto il pianeta è quello che è sulla bocca di tutti, ovvero il controllo delle emissioni di gas serra.
		
		Tuttavia, questo tipo di soluzione comporta prese di posizione e interventi che probabilmente non riusciranno mai ad essere attuati, avendo una fortissima connotazione politico-strategica che coinvolge molti aspetti dell'economia. E comunque, anche se le emissioni si arrestassero oggi, i cambiamenti climatici continuerebbero a
		lungo a causa dell'accumulo storico di gas a effetto serra nell'atmosfera: si otterrebbe comunque una ``attenuazione'' dei cambiamenti climatici evitando in parte gli impatti ingestibili dei cambiamenti climatici.
		
		In questa prospettiva, è importante cominciare mettere in atto misure di adattamento \cite{Relazione_AEA}, valutando e affrontando la
		vulnerabilità dei sistemi naturali e antropogeni a impatti quali alluvioni, siccità, innalzamento del livello del mare, malattie e ondate di calore, riconsiderando dove e come viviamo ora e dove e come vivremo in futuro.	
		
		\subsection{Fattori di successo}
		
		Il sostegno politico, elaborando un quadro strategico di riferimento all'interno del quale svolgere una azione efficace, rappresenta un catalizzatore fondamentale per avviare, spingere e coordinare l'adattamento al cambiamento climatico. 
		
		Le politiche adottate sono in genere reazioni a eventi estremi o a calamità naturali che spingono a richiedere l'intervento delle autorità pubbliche. Evidentemente sarebbe opportuna una maggiore lungimiranza, con misure che vadano oltre le contingenze e riescano a fornire strumenti preventivi efficaci: ma quali dovrebbero essere le caratteristiche delle misure adottate affinché si rivelino tali?
		
		\begin{itemize}
			\item Gli interventi sono generalmente più accettati e riusciti quando \textbf{promuovono} (o per lo meno non contrastano) \textbf{altri obiettivi}: il
			profitto economico delle aziende private coinvolte a vario titolo nelle scelte adottate è certamente il più significativo tra questi.
			\item Un \textbf{solido quadro giuridico} è un importante contributo al sostegno politico. Può definire un mandato chiaro per creare cooperazione a livello del bacino idrico, su scala intersettoriale
			o interregionale, facilitando la condivisione delle risorse idriche e il coordinamento degli utilizzatori di acqua e terreno.
			\item Gli \textbf{interventi di carattere tecnologico} (il miglioramento delle tecniche di irrigazione, nuovi bacini idrici, raccolta dell'acqua piovana, reimpiego delle acque reflue e delle acque
			grigie) svolgono un ruolo fondamentale negli interventi di adattamento.
			\item Un crescente numero di progetti 	azioni complementari ``soft'' relative alla \textbf{gestione della domanda}, come l'adozione di comportamenti diversi e la piena partecipazione e assunzione di responsabilità dei vari attori. 
			\item Anche l'introduzione di \textbf{incentivi economici} di mercato (in particolare la determinazione del prezzo dell'acqua) e di sostegni finanziari (per esempio sussidi) risulta utile per favorire interventi di adattamento attivi e innovativi,
			che garantiscono la partecipazione del settore privato e aumentano le probabilità di successo delle misure adottate.
		\end{itemize}
		
	\subsection{Ostacoli all'adattamento}
		La letteratura e i casi di studio in essa contenuti hanno evidenziato, senza peraltro sorprendere, che esistono alcuni fattori i quali, se non accuratamente tenuti in considerazione, riducono la capacità di sviluppare misure di adattamento au cambiamenti climatici.
		
		\begin{itemize}
		\item \textbf{la limitata conoscenza scientifica e l'incertezza} circa le conseguenze dei futuri cambiamenti climatici sulla disponibilità, la qualità e la domanda di acqua a livello locale è chiaramente un ostacolo fondamentale all'impegno politico per intervenire in anticipo e in maniera lungimirante a favore dell'adattamento. In parte, ciò è dovuto alle enormi incertezze nel downscaling (abbassamento di scala) di scenari e modelli climatici;
		\item \textbf{l'assenza di strategie di pianificazione di lungo periodo, di coordinamento e dell'uso degli strumenti gestionali} che affrontano il cambiamento globale a livello regionale, intersettoriale e dei bacini fluviali impedisce
		uno sviluppo sostenibile delle risorse idriche e costituisce un ostacolo fondamentale all'elaborazione di efficaci politiche di adattamento. In genere, le reti idriche che  collegano comunità o regioni possono supplire meglio alle carenze locali e contribuire a evitare azioni scoordinate e soluzioni individualiste o inefficienti;
		\item \textbf{il cambiamento climatico viene raramente considerato in modo esplicito }nei piani di gestione della domanda e offerta di acqua, mancano perciò interventi di adattamento in
		campo idrico che reagiscano specificamente alle	conseguenze attuali e future del cambiamento climatico. Ciò è dovuto in parte alle scarse conoscenze sull'impatto locale e regionale del cambiamento climatico. La Direttiva quadro in materia di acque e i Piani di gestione dei distretti idrografici (i primi devono essere	elaborati nel 2010) dovrebbero sicuramente contribuire all'ottimizzazione dell'adattamento
		al cambiamento climatico e al suo inserimento nelle politiche settoriali.
		\end{itemize}
		
		\subsection{Le opzioni politiche possibili}
		
		Alla luce degli studi e dei modelli analizzati in letteratura, è importante che vengano elaborate delle politiche efficaci per l'adattamento
		ai cambiamenti climatici e ai problemi idrici.
		
		\paragraph{La necessità di strategie di adattamento regionali e	locali:}
		
		i diversi attori devono operare in modo trasversale
		con comuni, aree montane o bacini/valli fluviali
		per affrontare in maniera efficace le conseguenze
		del cambiamento climatico e le questioni relative
		all'adattamento. Strategie locali e regionali
		dovrebbero prendere in considerazione, per quanto
		possibile, le caratteristiche della situazione locale
		(tipologia e tempistica dei problemi idrici;
		circostanze economiche; servizi ecosistemici; fattori
		che favoriscono l'adattamento, elementi di successo
		e ostacoli).
		
		La natura intersettoriale delle risorse idriche e il loro essere intrinsecamente transnazionali richiede anche
		approcci integrati e una \textit{governance} a più livelli
		che possono contribuire al coordinamento degli
		attori provenienti da diversi settori, regioni e livelli
		politici (locale, regionale, nazionale, UE).
		
		Le strategie di adattamento regionale devono
		assicurare uno scambio di informazioni coordinato
		ed efficiente tra i diversi livelli politici e i vari
		attori in modo che il cambiamento climatico trovi
		un'adeguata espressione nelle relative politiche
		(``climate proofing''). Esistono alcuni strumenti
		gestionali che considerano il cambiamento climatico
		(strutture analitiche, scenari adattati a scala
		locale, analisi costi/benefici, esempi di \textit{good practice})ma essi restano in gran parte ignoti agli attori locali e regionali, il che evidenzia che vi è un'elevata
		necessità di campagne di diffusione e informazione, volta anche a coinvolgere la cittadinanza e renderla attore consapevole e non solo spettatore passivo. Peraltro, come visto, l'apporto della cittadinanza è fondamentale in quanto le amministrazioni, da sole, non possono affrontare la questione con grandi speranza di successo.
		
		Le decisioni relative agli adattamenti attuali e futuri 	al cambiamento climatico a livello locale e regionale sono necessariamente assunte in condizioni di incertezza. \`{E} quindi necessaria una gestione adattiva, dotata di un rigido monitoraggio, per consentire una revisione regolare degli obiettivi politici e l'integrazione di nuove informazioni scientifiche che, non appena rese disponibili, possono ridurre l'incertezza. Da questo punto di vista, 'adattamento al cambiamento climatico non è diverso da qualsiasi altra questione ambientale dove mancano informazioni precise e una completa competenza scientifica. Il principio di precauzione
		è una risposta adeguata alle carenze conoscitive. 
		
		Secondo il principio di precauzione, l'assenza di una piena certezza scientifica non dovrebbe essere utilizzata come motivo per posticipare qualsiasi intervento laddove esista il rischio di danno grave o irreversibile alla salute pubblica o all'ambiente.
		
		\textbf{Per costruire una capacità di resistenza nei
		confronti di eventi inattesi} si deve prendere in
		considerazione anche la possibilità di cambiamenti
		non lineari, improvvisi o irregolari, in grado
		di modificare lo stato dell'ambiente una volta
		raggiunta una determinata soglia.
				
		
		\paragraph{Le strategie di adattamento devono riguardare tutti i settori:} le risorse idriche costituiscono una questione intrinsecamente multi-disciplinare, che riguarda
		praticamente tutti i settori socio-economici.
		
		I principali settori che esercitano pressioni sulle risorse idriche (in relazione alla quantità e alla qualità dell'acqua) sono agricoltura, consumi domestici, turismo e produzione di energia, mentre i principali settori socio-economici e ambientali colpiti dalla concorrenza per le risorse idriche sono agricoltura, conservazione della biodiversità, consumi domestici e produzione energetica.
		
		Durante l'elaborazione e l'attuazione degli interventi di adattamento si dovrebbe utilizzare un approccio di \textbf{gestione integrata delle risorse idriche}. Tali interventi devono solitamente prendere in considerazione i diversi elementi
		che favoriscono la riuscita oppure ostacolano l'adattamento. Secondo l'approccio integrato, i vari fattori (un ambiente favorevole, la definizione di ruoli e funzioni istituzionali e l'impiego di strumenti gestionali) devono essere coerentemente e costantemente integrati nella gestione delle
		risorse idriche. Una approfondita valutazione della concorrenza intersettoriale è importante per definire strategie di adattamento efficaci, proporzionate e trasparenti, e le varie opzioni su scala regionale. Tutte le strategie dovrebbero preferibilmente servirsi di un quadro di riferimento per i beni e servizi ecosistemici.
		
		\paragraph{L'Unione Europea dovrebbe fornire il quadro
		politico di riferimento:} le dimensioni europee degli impatti del cambiamento climatico e dei problemi idrici
		nelle Alpi sono variegate. 
		
		Le iniziative della UE dovrebbero fornire ai vari attori la cornice e gli strumenti politici per affrontare in modo efficace le conseguenze del cambiamento climatico e
		sviluppare a livello nazionale e subnazionale
		strategie di adattamento che affrontino le problematiche locali. Ciò aiuterà a integrare il cambiamento climatico nelle politiche e ad accelerare e coordinare i progetti di
		adattamento in tutta Europa.
		
		Gli impatti del cambiamento climatico variano da
		regione a regione, con una particolare vulnerabilità
		delle zone montane (insieme alle aree costiere e
		alle pianure alluvionali). Per questo motivo gli
		interventi di adattamento troveranno in maggior
		parte attuazione a livello nazionale, regionale o
		locale. L'Unione Europea dovrebbe sostenere questi
		sforzi con un approccio coordinato e integrato,
		in particolare in riferimento alle questioni
		transnazionali e di solidarietà regionale, oltre
		che agli ambiti politici della UE (le politiche
		comunitarie, come agricoltura, acqua, biodiversità,
		pesca ed energia, e il mercato unico). 
		
		L'adattamento al cambiamento climatico dovrà essere inserito in tutte le politiche UE.
				
	\newpage
	
	\bibliographystyle{apalike}
	\bibliography{ecosystemServices.bib}
%	\newpage
%	\appendix
%	\section*{Note sullo sviluppo della discussione}
%	Dal generale al particolare:
%	\begin{itemize}
%		\item definizione di servizi ecosistemici (30 secondi)
%		\item trattazione generale dei servizi ecosistemici (1 minuto e 30 secondi);
%		\item questione delle alpi (30 secondi);
%		\item problemi relativi all'approvvigionamento idrico livello europeo (30 secondi);
%		\item situazione dei ghiacci (2 minuti);
%		\item prospettive  (2 minuti).
%	\end{itemize}
%
\end{document}
