% This is definitely *NOT* a thesis
\documentclass[11pt,a4paper]{article}
\usepackage[italian]{babel}
\usepackage[utf8]{inputenc}
\usepackage{graphicx}
\usepackage{url}
\usepackage[hidelinks]{hyperref}
\usepackage[big,binding=0cm]{layaureo}
\author{Alberto Azzalini}
\title{Valutazione dei servizi ecosistemici in ambiente alpino:\\ problemi e prospettive}
\begin{document}
	\maketitle
	\begin{abstract}
		L'introduzione del concetto di servizi ecosistemici ha permesso di dare un valore economico alle risorse naturali e ai benefici che esse apportano alla società umana: collegare in termini monetari aspetti biofisici e benessere dell'uomo si è rivelato un passo essenziale per mettere chiaramente in luce la dipendenza delle società umane dagli ecosistemi naturali.
		Nel passato, le società non sono riuscite a valutare l'importanza degli ecosistemi, che sono stati spesso ritenuti un bene di tutti e il cui valore è stato di conseguenza sottostimato, se non addirittura ritenuto trascurabile.
		
		La perdita dei servizi forniti dagli ecosistemi naturali comporterà la necessità di trovare alternative dispendiose. Gli investimenti nel nostro capitale naturale consentiranno di risparmiare nel lungo periodo e per questo sono essenziali per il nostro benessere e per la sopravvivenza a lungo termine. \cite{Ecosystem_goods_and_services}
		
		
	\end{abstract}
	\section{Monetizzare il valore di un ecosistema}

	[ECOSCIENZA Numero 3 Anno 2010]
	
	
	Avere una buona dotazione di servizi
	ecosistemici significa avere una maggior
	``ricchezza'' pro-capite in termini di
	capitale naturale, ma anche una minore
	vulnerabilità, una maggiore salute e
	resilienza dei territori.
	
	Ecosistemi sani possono offrire un contributo
	molto significativo, proprio perché i loro servizi, gratuitamente utilizzati dall'uomo, che costituiscono risorse	non sostituibili con quelle del capitale
	antropico, rappresentano un importante
	fattore economico, attualmente ignorato
	dall'economia tradizionale perché senza
	mercato, ma di importanza strategica in
	un'ottica di ecologia economica e di nuovi
	indicatori di integrazione del Pil (\url{www.stiglitz-sen-fitoussi.fr}, v. anche Ecoscienza
	2/2010 \url{www.ecoscienza.eu}).
	
	L'economia ecologica individua un
	nuovo approccio per pesare le risorse
	di un territorio e per riequilibrare i
	sistemi economici. Dal capitale inteso in senso generico e tradizionale viene enucleato quello ``naturale'', che fornisce naturalmente servizi mantenendo la stabilità ecologica dei sistemi,
	valorizzando i territori ricchi di SE e le
	attività economiche compatibili che ne
	concorrono a mantenere la funzionalità
	(es. agricoltura biologica). 
	
	La valutazione
	ecologica-economica ha lo scopo di
	stimare i SE in termini monetari, al
	fine di fornire una metrica comune
	attraverso cui i benefici di diversi servizi
	forniti dagli ecosistemi possano essere
	quantificati (MEA, 2005) al fine di
	supportare strategie di sostenibilità e di
	perequazione territoriale, anche a fronte
	dei cambiamenti globali nel breve, medio
	e lungo periodo. È importante quindi
	valutare il valore economico totale (TEV)
	(Freeman, 1993; Merlo e Croitoru, 2005;
	Dziegielewska et al., 2009), delle risorse
	e dei servizi considerati anche come
	beni pubblici considerandoli nelle analisi
	costi-benefici e spesso nelle valutazioni
	del danno ambientale, in cui il concetto
	di TEV costituisce il background
	metodologico delle valutazioni dei beni
	ambientali alla cui base c'è la distinzione
	tra due grandi categorie di benefici che
	una risorsa naturale offre: i valori d'uso
	e i valori di non-uso. 
	
	Conoscere il valore
	economico totale delle risorse e dei beni
	ambientali è quindi importante per
	verificare la razionalità delle scelte di
	sviluppo, per dare un valore alle politiche
	di tutela dell'ambiente e individuare le
	regioni più fragili dove il cambiamento è
	più probabile. Occorre quindi innescare
	dei meccanismi di riconoscimento
	economico di questi servizi (Santolini,
	2008) in modo che vengano pesati nel
	bilancio economico complessivo mediante
	un sistema metrico comune che faciliti
	le analisi dal momento che ora esistono
	metodi discussi ma efficaci di valutazione
	economico-ambientale (Giupponi et al.,
	2009) come già sviluppato a vari livelli sia
	locale (Morri e Santolini, 2009, 2010) che
	nazionale (Cataldi et al., 2009; Scolozzi et
	al., 2010).
	
	Gli attuali strumenti di pianificazione,
	pur nelle significative differenze,
	partono da un'analisi dello status delle
	risorse ambientali, spesso trascurando
	i processi ecosistemici, le interazioni
	dinamiche e di controllo dei processi
	stessi, in particolare le loro relazioni
	con i fattori economici e sociali. Inoltre
	la pianificazione di tipo settoriale (es.
	agricoltura-Psr, infrastrutture-piano della
	mobilità, gestione delle acque-piano delle
	acque, etc.) non è nei fatti coordinata,
	anche in seguito a una suddivisione di
	responsabilità tra entità amministrative,
	per esempio tra i livelli regionali e quelli
	locali, anche se questo dovrebbe essere
	oggetto della pianificazione strategica.
	
	Il paradigma dei SE può costituire
	quindi la base per una revisione dei
	termini economici con cui considerare il
	territorio e i suoi capitali attraverso una
	pianificazione territoriale più consapevole
	del significato dei processi ecologici e più
	orientata verso una sostenibilità concreta
	e durevole.
	
	
	\section{Conseguenze dei cambiamenti climatici per l'ecosistema alpino \cite{LeAlpi}}
	
	L'impatto dei cambiamenti climatici sui servizi ecosistemici alpini non si limita agli effetti sulla disponibilità di acqua potabile. Per ogni grado di aumento della temperatura, il livello della neve si alza di circa 150 metri. Di conseguenza, si accumula meno neve a bassa quota. Quasi la metà delle stazioni sciistiche in Svizzera, e un numero ancora maggiore in Germania, Austria e nei Pirenei, in futuro avranno difficoltà ad attirare turisti e amanti degli sport invernali.
	
	Anche le specie vegetali migrano verso nord e verso altitudini maggiori. Le cosiddette ``specie pionieristiche'' si spostano in altezza. Le piante che si sono adattate al freddo vengono ora cacciate dai loro habitat naturali. Entro la fine del XXI secolo le specie vegetali europee potrebbero spostarsi centinaia di chilometri a nord e il 60 per cento delle specie vegetali montane rischia l'estinzione.
	
	La riduzione osservata e prevista del permafrost verosimilmente aumenterà le calamità naturali e i danni alle infrastrutture presenti in alta quota. L'ondata di calore che ha colpito l'Europa nel 2003 ha evidenziato le conseguenze potenzialmente gravi dell'aumento delle temperature e della siccità sul benessere umano e sui settori economici che si basano sull'uso di acqua (come la produzione di elettricità). In quel solo anno lo scioglimento dei ghiacci ha ridotto di un terzo la massa dei ghiacciai alpini e ha provocato decine di migliaia di morti in Europa.
	
	Le Alpi presentano in anteprima le sfide con cui dovranno misurarsi gli ecosistemi, gli habitat e le popolazioni in tutta Europa e nel mondo. In un racconto sull'Artico, narrato nelle pagine seguenti, ascolteremo le persone che vivono nell'Europa artica descrivere gli effetti che i cambiamenti climatici stanno già esercitando sulla loro vita.
	
	\newpage
	
	\bibliographystyle{plain}
	\bibliography{ecosystemServices}
\end{document}
