% This is definitely *NOT* a thesis
\documentclass[11pt,a4paper]{article}
\usepackage[italian]{babel}
\usepackage[utf8]{inputenc}
\usepackage{graphicx}
\usepackage{url}
\usepackage[hidelinks]{hyperref}
\usepackage[big,binding=0cm]{layaureo}
\author{Alberto Azzalini}
\title{Valutazione dei servizi ecosistemici in ambiente alpino:\\ problemi e prospettive}
\begin{document}
	\maketitle
	\begin{abstract}
		L'introduzione del concetto di servizi ecosistemici ha permesso di dare un valore economico alle risorse naturali e ai benefici che esse apportano alla società umana: collegare in termini monetari aspetti biofisici e benessere dell'uomo si è rivelato un passo essenziale per mettere chiaramente in luce la dipendenza delle società umane dagli ecosistemi naturali.
		Nel passato, le società non sono riuscite a valutare l'importanza degli ecosistemi, che sono stati spesso ritenuti un bene di tutti e il cui valore è stato di conseguenza sottostimato, se non addirittura ritenuto trascurabile.
		
		La perdita dei servizi forniti dagli ecosistemi naturali comporterà la necessità di trovare alternative dispendiose. Gli investimenti nel nostro capitale naturale consentiranno di risparmiare nel lungo periodo e per questo sono essenziali per il nostro benessere e per la sopravvivenza a lungo termine. \cite{Ecosystem_goods_and_services}
		
		
	\end{abstract}
	\section{Monetizzare il valore di un ecosistema}
	
	
	
	\section{Conseguenze dei cambiamenti climatici per l'ecosistema alpino \cite{LeAlpi}}
	
	L'impatto dei cambiamenti climatici sui servizi ecosistemici alpini non si limita agli effetti sulla disponibilità di acqua potabile. Per ogni grado di aumento della temperatura, il livello della neve si alza di circa 150 metri. Di conseguenza, si accumula meno neve a bassa quota. Quasi la metà delle stazioni sciistiche in Svizzera, e un numero ancora maggiore in Germania, Austria e nei Pirenei, in futuro avranno difficoltà ad attirare turisti e amanti degli sport invernali.
	
	Anche le specie vegetali migrano verso nord e verso altitudini maggiori. Le cosiddette “specie pionieristiche” si spostano in altezza. Le piante che si sono adattate al freddo vengono ora cacciate dai loro habitat naturali. Entro la fine del XXI secolo le specie vegetali europee potrebbero spostarsi centinaia di chilometri a nord e il 60 per cento delle specie vegetali montane rischia l'estinzione.
	
	La riduzione osservata e prevista del permafrost verosimilmente aumenterà le calamità naturali e i danni alle infrastrutture presenti in alta quota. L'ondata di calore che ha colpito l'Europa nel 2003 ha evidenziato le conseguenze potenzialmente gravi dell'aumento delle temperature e della siccità sul benessere umano e sui settori economici che si basano sull'uso di acqua (come la produzione di elettricità). In quel solo anno lo scioglimento dei ghiacci ha ridotto di un terzo la massa dei ghiacciai alpini e ha provocato decine di migliaia di morti in Europa.
	
	Le Alpi presentano in anteprima le sfide con cui dovranno misurarsi gli ecosistemi, gli habitat e le popolazioni in tutta Europa e nel mondo. In un racconto sull'Artico, narrato nelle pagine seguenti, ascolteremo le persone che vivono nell'Europa artica descrivere gli effetti che i cambiamenti climatici stanno già esercitando sulla loro vita.
	
	\newpage
	
	\bibliographystyle{plain}
	\bibliography{ecosystemServices}
\end{document}
